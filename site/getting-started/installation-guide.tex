% Options for packages loaded elsewhere
% Options for packages loaded elsewhere
\PassOptionsToPackage{unicode,bookmarksopen=true,bookmarksnumbered=true}{hyperref}
\PassOptionsToPackage{hyphens}{url}
\PassOptionsToPackage{dvipsnames,svgnames,x11names}{xcolor}
%
\documentclass[
  12pt,
  letterpaper,
  oneside]{book}
\usepackage{xcolor}
\usepackage[margin=1in]{geometry}
\usepackage{amsmath,amssymb}
\setcounter{secnumdepth}{5}
\usepackage{iftex}
\ifPDFTeX
  \usepackage[T1]{fontenc}
  \usepackage[utf8]{inputenc}
  \usepackage{textcomp} % provide euro and other symbols
\else % if luatex or xetex
  \usepackage{unicode-math} % this also loads fontspec
  \defaultfontfeatures{Scale=MatchLowercase}
  \defaultfontfeatures[\rmfamily]{Ligatures=TeX,Scale=1}
\fi
\usepackage{lmodern}
\ifPDFTeX\else
  % xetex/luatex font selection
  \setmainfont[]{Times New Roman}
  \setsansfont[]{Arial}
\fi
% Use upquote if available, for straight quotes in verbatim environments
\IfFileExists{upquote.sty}{\usepackage{upquote}}{}
\IfFileExists{microtype.sty}{% use microtype if available
  \usepackage[]{microtype}
  \UseMicrotypeSet[protrusion]{basicmath} % disable protrusion for tt fonts
}{}
\usepackage{setspace}
\makeatletter
\@ifundefined{KOMAClassName}{% if non-KOMA class
  \IfFileExists{parskip.sty}{%
    \usepackage{parskip}
  }{% else
    \setlength{\parindent}{0pt}
    \setlength{\parskip}{6pt plus 2pt minus 1pt}}
}{% if KOMA class
  \KOMAoptions{parskip=half}}
\makeatother
% Make \paragraph and \subparagraph free-standing
\makeatletter
\ifx\paragraph\undefined\else
  \let\oldparagraph\paragraph
  \renewcommand{\paragraph}{
    \@ifstar
      \xxxParagraphStar
      \xxxParagraphNoStar
  }
  \newcommand{\xxxParagraphStar}[1]{\oldparagraph*{#1}\mbox{}}
  \newcommand{\xxxParagraphNoStar}[1]{\oldparagraph{#1}\mbox{}}
\fi
\ifx\subparagraph\undefined\else
  \let\oldsubparagraph\subparagraph
  \renewcommand{\subparagraph}{
    \@ifstar
      \xxxSubParagraphStar
      \xxxSubParagraphNoStar
  }
  \newcommand{\xxxSubParagraphStar}[1]{\oldsubparagraph*{#1}\mbox{}}
  \newcommand{\xxxSubParagraphNoStar}[1]{\oldsubparagraph{#1}\mbox{}}
\fi
\makeatother


\usepackage{longtable,booktabs,array}
\usepackage{calc} % for calculating minipage widths
% Correct order of tables after \paragraph or \subparagraph
\usepackage{etoolbox}
\makeatletter
\patchcmd\longtable{\par}{\if@noskipsec\mbox{}\fi\par}{}{}
\makeatother
% Allow footnotes in longtable head/foot
\IfFileExists{footnotehyper.sty}{\usepackage{footnotehyper}}{\usepackage{footnote}}
\makesavenoteenv{longtable}
\usepackage{graphicx}
\makeatletter
\newsavebox\pandoc@box
\newcommand*\pandocbounded[1]{% scales image to fit in text height/width
  \sbox\pandoc@box{#1}%
  \Gscale@div\@tempa{\textheight}{\dimexpr\ht\pandoc@box+\dp\pandoc@box\relax}%
  \Gscale@div\@tempb{\linewidth}{\wd\pandoc@box}%
  \ifdim\@tempb\p@<\@tempa\p@\let\@tempa\@tempb\fi% select the smaller of both
  \ifdim\@tempa\p@<\p@\scalebox{\@tempa}{\usebox\pandoc@box}%
  \else\usebox{\pandoc@box}%
  \fi%
}
% Set default figure placement to htbp
\def\fps@figure{htbp}
\makeatother





\setlength{\emergencystretch}{3em} % prevent overfull lines

\providecommand{\tightlist}{%
  \setlength{\itemsep}{0pt}\setlength{\parskip}{0pt}}



 


\makeatletter
\@ifpackageloaded{caption}{}{\usepackage{caption}}
\AtBeginDocument{%
\ifdefined\contentsname
  \renewcommand*\contentsname{Table of contents}
\else
  \newcommand\contentsname{Table of contents}
\fi
\ifdefined\listfigurename
  \renewcommand*\listfigurename{List of Figures}
\else
  \newcommand\listfigurename{List of Figures}
\fi
\ifdefined\listtablename
  \renewcommand*\listtablename{List of Tables}
\else
  \newcommand\listtablename{List of Tables}
\fi
\ifdefined\figurename
  \renewcommand*\figurename{Figure}
\else
  \newcommand\figurename{Figure}
\fi
\ifdefined\tablename
  \renewcommand*\tablename{Table}
\else
  \newcommand\tablename{Table}
\fi
}
\@ifpackageloaded{float}{}{\usepackage{float}}
\floatstyle{ruled}
\@ifundefined{c@chapter}{\newfloat{codelisting}{h}{lop}}{\newfloat{codelisting}{h}{lop}[chapter]}
\floatname{codelisting}{Listing}
\newcommand*\listoflistings{\listof{codelisting}{List of Listings}}
\makeatother
\makeatletter
\makeatother
\makeatletter
\@ifpackageloaded{caption}{}{\usepackage{caption}}
\@ifpackageloaded{subcaption}{}{\usepackage{subcaption}}
\makeatother
\usepackage{bookmark}
\IfFileExists{xurl.sty}{\usepackage{xurl}}{} % add URL line breaks if available
\urlstyle{same}
\hypersetup{
  pdftitle={Installation Guide},
  colorlinks=true,
  linkcolor={Maroon},
  filecolor={Maroon},
  citecolor={Blue},
  urlcolor={Blue},
  pdfcreator={LaTeX via pandoc}}


\title{Installation Guide}
\author{}
\date{}
\begin{document}
\frontmatter
\maketitle

\renewcommand*\contentsname{Table of contents}
{
\hypersetup{linkcolor=}
\setcounter{tocdepth}{2}
\tableofcontents
}

\setstretch{1.5}
\mainmatter
This guide provides detailed, OS-specific steps for installing the core
software needed for this course: R, RStudio, and Git.

\begin{center}\rule{0.5\linewidth}{0.5pt}\end{center}

\chapter{Installing on Windows}\label{installing-on-windows}

Follow these steps in order.

\section{1. Install R}\label{install-r}

\begin{enumerate}
\def\labelenumi{\arabic{enumi}.}
\tightlist
\item
  Go to the \href{https://cran.r-project.org/bin/windows/base/}{CRAN
  (Comprehensive R Archive Network)}.
\item
  Click the large \textbf{``Download R-X.X.X for Windows''} link at the
  top of the page.
\item
  Run the downloaded installer (\texttt{.exe} file).
\item
  Accept all the default settings during installation. Click ``Next''
  until it's finished.
\end{enumerate}

\section{2. Install RStudio}\label{install-rstudio}

\begin{enumerate}
\def\labelenumi{\arabic{enumi}.}
\tightlist
\item
  Go to the \href{https://posit.co/download/rstudio-desktop/}{Posit
  website}.
\item
  Click the ``Download RStudio Desktop'' button. The site should
  automatically detect you are on Windows.
\item
  Run the downloaded installer.
\item
  Accept all the default settings.
\end{enumerate}

\section{3. Install Git}\label{install-git}

\begin{enumerate}
\def\labelenumi{\arabic{enumi}.}
\tightlist
\item
  Go to the \href{https://git-scm.com/downloads}{official Git website}.
\item
  Click the ``Windows'' link to download the installer.
\item
  Run the installer. This one has many screens, but you can safely
  \textbf{accept all the default options}. The most important thing is
  to ensure that Git is added to your system's PATH, which is the
  default behavior.
\end{enumerate}

\begin{center}\rule{0.5\linewidth}{0.5pt}\end{center}

\chapter{Installing on macOS}\label{installing-on-macos}

Follow these steps in order.

\section{1. Install R}\label{install-r-1}

\begin{enumerate}
\def\labelenumi{\arabic{enumi}.}
\tightlist
\item
  Go to the \href{https://cran.r-project.org/bin/macosx/}{CRAN
  (Comprehensive R Archive Network)}.
\item
  Look for the ``Latest release'' section. Download the package
  (\texttt{.pkg} file) that matches your Mac's processor (Apple Silicon
  ``M1/M2/M3'' or Intel). If you're unsure, click the Apple menu →
  ``About This Mac'' to check.
\item
  Run the downloaded installer. Accept all default settings.
\end{enumerate}

\section{2. Install RStudio}\label{install-rstudio-1}

\begin{enumerate}
\def\labelenumi{\arabic{enumi}.}
\tightlist
\item
  Go to the \href{https://posit.co/download/rstudio-desktop/}{Posit
  website}.
\item
  Click the ``Download RStudio Desktop'' button. The site should
  automatically detect you are on macOS.
\item
  This will download a \texttt{.dmg} file. Open it.
\item
  A new window will appear. Drag the RStudio icon into the Applications
  folder icon.
\item
  The first time you open RStudio, you may get a security warning. Click
  ``Open'' to proceed.
\end{enumerate}

\section{3. Install Git}\label{install-git-1}

macOS comes with Git pre-installed. However, it's often best to install
a newer version using \href{https://brew.sh/}{Homebrew}, a package
manager for macOS.

\begin{enumerate}
\def\labelenumi{\arabic{enumi}.}
\tightlist
\item
  \textbf{Install Homebrew (if you don't have it):} Open the
  \textbf{Terminal} app (you can find it in Applications/Utilities) and
  paste the following command, then press Enter:
  \texttt{bash\ \ \ \ \ /bin/bash\ -c\ "\$(curl\ -fsSL\ https://raw.githubusercontent.com/Homebrew/install/HEAD/install.sh)"}
\item
  \textbf{Install Git:} Once Homebrew is installed, run the following
  command in the Terminal: \texttt{bash\ \ \ \ \ brew\ install\ git}
  This ensures you have an up-to-date version of Git.
\end{enumerate}


\backmatter


\end{document}
