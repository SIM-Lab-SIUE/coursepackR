% Options for packages loaded elsewhere
% Options for packages loaded elsewhere
\PassOptionsToPackage{unicode,bookmarksopen=true,bookmarksnumbered=true}{hyperref}
\PassOptionsToPackage{hyphens}{url}
\PassOptionsToPackage{dvipsnames,svgnames,x11names}{xcolor}
%
\documentclass[
  12pt,
  letterpaper,
  oneside]{book}
\usepackage{xcolor}
\usepackage[margin=1in]{geometry}
\usepackage{amsmath,amssymb}
\setcounter{secnumdepth}{5}
\usepackage{iftex}
\ifPDFTeX
  \usepackage[T1]{fontenc}
  \usepackage[utf8]{inputenc}
  \usepackage{textcomp} % provide euro and other symbols
\else % if luatex or xetex
  \usepackage{unicode-math} % this also loads fontspec
  \defaultfontfeatures{Scale=MatchLowercase}
  \defaultfontfeatures[\rmfamily]{Ligatures=TeX,Scale=1}
\fi
\usepackage{lmodern}
\ifPDFTeX\else
  % xetex/luatex font selection
  \setmainfont[]{Times New Roman}
  \setsansfont[]{Arial}
\fi
% Use upquote if available, for straight quotes in verbatim environments
\IfFileExists{upquote.sty}{\usepackage{upquote}}{}
\IfFileExists{microtype.sty}{% use microtype if available
  \usepackage[]{microtype}
  \UseMicrotypeSet[protrusion]{basicmath} % disable protrusion for tt fonts
}{}
\usepackage{setspace}
\makeatletter
\@ifundefined{KOMAClassName}{% if non-KOMA class
  \IfFileExists{parskip.sty}{%
    \usepackage{parskip}
  }{% else
    \setlength{\parindent}{0pt}
    \setlength{\parskip}{6pt plus 2pt minus 1pt}}
}{% if KOMA class
  \KOMAoptions{parskip=half}}
\makeatother
% Make \paragraph and \subparagraph free-standing
\makeatletter
\ifx\paragraph\undefined\else
  \let\oldparagraph\paragraph
  \renewcommand{\paragraph}{
    \@ifstar
      \xxxParagraphStar
      \xxxParagraphNoStar
  }
  \newcommand{\xxxParagraphStar}[1]{\oldparagraph*{#1}\mbox{}}
  \newcommand{\xxxParagraphNoStar}[1]{\oldparagraph{#1}\mbox{}}
\fi
\ifx\subparagraph\undefined\else
  \let\oldsubparagraph\subparagraph
  \renewcommand{\subparagraph}{
    \@ifstar
      \xxxSubParagraphStar
      \xxxSubParagraphNoStar
  }
  \newcommand{\xxxSubParagraphStar}[1]{\oldsubparagraph*{#1}\mbox{}}
  \newcommand{\xxxSubParagraphNoStar}[1]{\oldsubparagraph{#1}\mbox{}}
\fi
\makeatother


\usepackage{longtable,booktabs,array}
\usepackage{calc} % for calculating minipage widths
% Correct order of tables after \paragraph or \subparagraph
\usepackage{etoolbox}
\makeatletter
\patchcmd\longtable{\par}{\if@noskipsec\mbox{}\fi\par}{}{}
\makeatother
% Allow footnotes in longtable head/foot
\IfFileExists{footnotehyper.sty}{\usepackage{footnotehyper}}{\usepackage{footnote}}
\makesavenoteenv{longtable}
\usepackage{graphicx}
\makeatletter
\newsavebox\pandoc@box
\newcommand*\pandocbounded[1]{% scales image to fit in text height/width
  \sbox\pandoc@box{#1}%
  \Gscale@div\@tempa{\textheight}{\dimexpr\ht\pandoc@box+\dp\pandoc@box\relax}%
  \Gscale@div\@tempb{\linewidth}{\wd\pandoc@box}%
  \ifdim\@tempb\p@<\@tempa\p@\let\@tempa\@tempb\fi% select the smaller of both
  \ifdim\@tempa\p@<\p@\scalebox{\@tempa}{\usebox\pandoc@box}%
  \else\usebox{\pandoc@box}%
  \fi%
}
% Set default figure placement to htbp
\def\fps@figure{htbp}
\makeatother





\setlength{\emergencystretch}{3em} % prevent overfull lines

\providecommand{\tightlist}{%
  \setlength{\itemsep}{0pt}\setlength{\parskip}{0pt}}



 


\makeatletter
\@ifpackageloaded{caption}{}{\usepackage{caption}}
\AtBeginDocument{%
\ifdefined\contentsname
  \renewcommand*\contentsname{Table of contents}
\else
  \newcommand\contentsname{Table of contents}
\fi
\ifdefined\listfigurename
  \renewcommand*\listfigurename{List of Figures}
\else
  \newcommand\listfigurename{List of Figures}
\fi
\ifdefined\listtablename
  \renewcommand*\listtablename{List of Tables}
\else
  \newcommand\listtablename{List of Tables}
\fi
\ifdefined\figurename
  \renewcommand*\figurename{Figure}
\else
  \newcommand\figurename{Figure}
\fi
\ifdefined\tablename
  \renewcommand*\tablename{Table}
\else
  \newcommand\tablename{Table}
\fi
}
\@ifpackageloaded{float}{}{\usepackage{float}}
\floatstyle{ruled}
\@ifundefined{c@chapter}{\newfloat{codelisting}{h}{lop}}{\newfloat{codelisting}{h}{lop}[chapter]}
\floatname{codelisting}{Listing}
\newcommand*\listoflistings{\listof{codelisting}{List of Listings}}
\makeatother
\makeatletter
\makeatother
\makeatletter
\@ifpackageloaded{caption}{}{\usepackage{caption}}
\@ifpackageloaded{subcaption}{}{\usepackage{subcaption}}
\makeatother
\usepackage{bookmark}
\IfFileExists{xurl.sty}{\usepackage{xurl}}{} % add URL line breaks if available
\urlstyle{same}
\hypersetup{
  colorlinks=true,
  linkcolor={Maroon},
  filecolor={Maroon},
  citecolor={Blue},
  urlcolor={Blue},
  pdfcreator={LaTeX via pandoc}}


\author{}
\date{}
\begin{document}
\frontmatter

\renewcommand*\contentsname{Table of contents}
{
\hypersetup{linkcolor=}
\setcounter{tocdepth}{2}
\tableofcontents
}

\setstretch{1.5}
\mainmatter
\chapter*{Building on Knowledge: The Literature
Review}\label{building-on-knowledge-the-literature-review}
\addcontentsline{toc}{chapter}{Building on Knowledge: The Literature
Review}

\section*{Entering the Scholarly
Conversation}\label{entering-the-scholarly-conversation}
\addcontentsline{toc}{section}{Entering the Scholarly Conversation}

Imagine you are walking into a room where a lively and complex
conversation has been going on for a long time. The participants are
knowledgeable and passionate, debating a topic from various angles,
building on each other's points, and challenging established ideas. You
have a new thought you are eager to share, an observation you believe is
important. But if you simply blurt it out without first listening to
what has already been said, your contribution will likely be ignored,
dismissed as naive, or seen as a repetition of a point made long ago. To
contribute meaningfully, you must first listen. You must understand the
history of the conversation, identify the key speakers, grasp the major
points of agreement and contention, and recognize what is currently
being discussed.

This is the perfect metaphor for the research process. No study is
conducted in a vacuum; it is part of an ongoing scholarly conversation
that has been unfolding for years, sometimes decades, across academic
journals, books, and conference papers. The \textbf{literature review}
is the essential and disciplined act of listening to that conversation.
For many, this is the most intimidating part of the research process,
feeling like a monumental task of finding, reading, and summarizing an
endless number of articles. This chapter aims to reframe that task. A
literature review is not a passive summary; it is an \textbf{active,
purposeful exploration} and a persuasive argument. It is the
intellectual labor of finding, evaluating, and synthesizing previous
scholarship to build a compelling case for your own work. Mastering this
process is how you earn the right to ask your question, transforming a
personal interest into a legitimate scholarly inquiry. This chapter
provides a detailed roadmap to this foundational act of scholarship,
demystifying the process and equipping you with both traditional and
contemporary strategies for success.

\section*{The Purpose and Goals of a Literature
Review}\label{the-purpose-and-goals-of-a-literature-review}
\addcontentsline{toc}{section}{The Purpose and Goals of a Literature
Review}

Before diving into the mechanics, it is essential to understand
\emph{why} the literature review is so fundamental. A well-executed
review is not just a hurdle to clear; it is a multi-faceted tool that
strengthens every aspect of your research project.

A primary goal of the literature review is \textbf{to situate your
research within an existing dialogue}. This demonstrates to your
audience that you are aware of the broader context and are not working
in isolation. By connecting your project to established theories and
previous findings, you are consciously building upon the collective
knowledge of your field rather than starting from scratch. This act of
positioning your study as the next logical step in a chain of inquiry
shows scholarly maturity and lends credibility to your work. It proves
you have done your homework and understand the landscape of knowledge
you seek to contribute to.

Perhaps the most crucial function of the literature review is \textbf{to
justify the need for your study by identifying a ``gap''} in the
existing body of work. This is how you answer the critical ``so what?''
question that every researcher must face. This ``gap'' is the compelling
rationale for your research, and it can take several forms. You may
identify a \textbf{topical void}, where no one has studied your specific
topic, population, or a new technology. You might uncover a
\textbf{contradiction}, where previous studies have produced conflicting
findings, creating an inconsistency that your work aims to resolve. Or,
you may propose an \textbf{alternative explanation}, where existing
theories seem insufficient, and you believe a new perspective could be
more insightful. By systematically demonstrating this gap, the
literature review persuades the reader that your study is not redundant
but is essential for advancing our collective understanding.

Furthermore, a thorough review serves the practical goal of helping you
\textbf{avoid ``reinventing the wheel''}. It is a frustrating but common
experience for a novice researcher to believe they have an original
idea, only to discover it was the subject of a dissertation ten years
ago. The literature review is a due diligence process that saves you
from wasting time and effort on a question that has already been
adequately answered. Beyond this, the review allows you to \textbf{learn
from the methodological successes and failures of others}. By examining
the methods sections of previous studies, you can discover reliable and
valid measurement scales, successful sampling strategies for
hard-to-reach populations, or innovative analytical techniques you can
adapt for your own project. Conversely, you can also learn from the
limitations other authors identify in their work, allowing you to design
your study to avoid those same pitfalls and thereby strengthen your
contribution.

Finally, the process of engaging with existing scholarship is often what
helps you \textbf{refine and focus your research question}. A research
interest often starts broad, such as a general curiosity about ``social
media and politics''. It is through reading the literature that you
discover the specific debates, concepts, terminology, and theoretical
frameworks that allow you to sharpen that interest into a precise,
researchable question. You might, for example, discover a nuanced debate
about the role of visual memes in fostering affective polarization among
young voters on Instagram, a far more specific and empirically
investigable topic than your initial idea. The literature provides the
tools to move from a vague interest to a focused scholarly inquiry.

\section*{The Literature Review Process: A Step-by-Step
Roadmap}\label{the-literature-review-process-a-step-by-step-roadmap}
\addcontentsline{toc}{section}{The Literature Review Process: A
Step-by-Step Roadmap}

The literature review becomes far more manageable when broken down into
a series of logical steps. This process moves from broad exploration to
a focused, written argument that serves as the foundation for your
research proposal.

\subsection*{Step 1: Topic Identification and Keyword
Generation}\label{step-1-topic-identification-and-keyword-generation}
\addcontentsline{toc}{subsection}{Step 1: Topic Identification and
Keyword Generation}

The process begins by translating your research topic into a set of
\textbf{keywords} that will be used to search academic databases. This
is a crucial brainstorming phase where you must think creatively and
expansively about your core concepts, generating a list of synonyms and
related terms for each. For instance, if your topic is the effect of
online news consumption on political polarization, your initial keyword
list might include:

\begin{itemize}
\item
  \textbf{Concept 1 (Online News):} ``online news,'' ``digital news,''
  ``internet news,'' ``social media news,'' ``news websites,'' ``news
  aggregators''
\item
  \textbf{Concept 2 (Political Polarization):} ``political
  polarization,'' ``partisan division,'' ``ideological extremity,''
  ``affective polarization,'' ``political disagreement''
\end{itemize}

Having a rich and varied list of keywords is essential because different
scholars may use different terminology to describe similar concepts.
This is not a one-time task; your keyword list should be a living
document. As you begin reading, you will discover the specific language
and jargon used in the scholarly literature on your topic, and you
should continuously update your list with these new terms.

\subsection*{Step 2: Strategically Searching for
Sources}\label{step-2-strategically-searching-for-sources}
\addcontentsline{toc}{subsection}{Step 2: Strategically Searching for
Sources}

With your initial keywords, you can begin the systematic search for
scholarly sources. A strategic, multi-pronged approach is far more
effective than a scattershot one. Your primary search arena will be your
university library's \textbf{academic databases}. These databases are
the gateway to peer-reviewed journal articles, which are considered the
``gold standard'' for scholarly research because their content has been
rigorously vetted by other experts in the field before publication.
While general-purpose search engines like Google Scholar are also
incredibly powerful, specialized databases like \emph{Communication \&
Mass Media Complete} or \emph{PsycINFO} provide more focused and curated
results for specific disciplines.

\begin{figure}[H]

{\centering \includegraphics[width=1\linewidth,height=\textheight,keepaspectratio]{images/boolean.jpg}

}

\caption{Visualization of Boolean Operator}

\end{figure}%

To search effectively, you must learn to combine your keywords with
\textbf{Boolean operators}. These simple commands refine your searches
dramatically.

\begin{itemize}
\item
  \textbf{AND} narrows your search by requiring both terms to appear
  (e.g., ``social media'' AND ``mental health'').
\item
  \textbf{OR} broadens your search by including synonyms, ensuring you
  don't miss relevant articles that use different terminology (e.g.,
  ``adolescents'' OR ``teenagers'').
\item
  \textbf{NOT} excludes irrelevant terms from your search (e.g.,
  ``social media'' NOT ``marketing'').
\end{itemize}

Perhaps the most powerful search strategy, however, is \textbf{citation
chaining}. Once you find one or two highly relevant ``keystone''
articles, you can use them to spiderweb out to the rest of the relevant
literature.

\textbf{Backward chaining} involves examining the reference list of your
keystone article. This is an excellent way to find the foundational and
seminal studies upon which the current research is built.

\textbf{Forward chaining} is the opposite; you use a tool like Google
Scholar to find your keystone article and click on the ``Cited by''
link. This reveals a list of all the subsequent articles that have cited
that work, which is the best way to bring your literature search up to
the present day and see how the scholarly conversation has evolved.

\begin{figure}[H]

{\centering \includegraphics[width=1\linewidth,height=\textheight,keepaspectratio]{images/forward-chain.png}

}

\caption{Example of Forward Chaining with Google Scholar}

\end{figure}%

\subsection*{Step 3: Navigating Information Overload and Evaluating
Credibility}\label{step-3-navigating-information-overload-and-evaluating-credibility}
\addcontentsline{toc}{subsection}{Step 3: Navigating Information
Overload and Evaluating Credibility}

In the digital age, the challenge is often not finding information, but
managing the overwhelming volume of it. Your initial searches will
likely yield hundreds or even thousands of potential sources. The next
step is to critically evaluate them to determine which are most relevant
and credible. Start by using the filters within academic databases to
narrow your results by publication date, methodology, or journal tier.
The most efficient way to quickly assess an article's relevance is to
\textbf{read the abstract first}. This concise summary of the study's
purpose, methods, and findings will tell you if the full article is
worth your time.

As you select sources, you must be a vigilant gatekeeper of quality,
especially given the rise of questionable publishing outlets. Prioritize
\textbf{peer-reviewed journal articles} and scholarly books from
reputable academic presses, as these have undergone the most rigorous
review process. Be particularly wary of \textbf{predatory journals},
which exploit the ``publish or perish'' pressure on academics by
charging publication fees without providing legitimate peer review. Red
flags include aggressive email solicitations, a suspiciously broad
scope, a poorly designed website, and an editorial board with
questionable credentials.

Furthermore, a critical evaluation extends to the content itself. Ask
yourself key questions as you skim articles:

\begin{itemize}
\item
  \textbf{Relevance:} How directly does this study address my specific
  research question? Is it a central piece of the puzzle or only
  tangentially related?
\item
  \textbf{Rigor:} Is the research design sound and the methodology
  clearly described? Is the journal reputable within your field?
\item
  \textbf{Currency:} When was this published? Is it a recent study
  reflecting the current state of the conversation, or is it an older,
  foundational piece that is still cited for its theoretical importance?
\end{itemize}

Finally, a note on \textbf{AI-assisted tools}: new AI technologies can
help generate keywords or summarize articles. While these can be useful
for initial exploration, they are not a substitute for your own critical
reading and analysis. AI summaries can be inaccurate, miss crucial
nuance, or even ``hallucinate'' information that isn't in the original
text. You must always read the original sources yourself to ensure a
correct and deep understanding. These tools are assistants, not
replacements for your scholarly judgment.

\begin{figure}[H]

{\centering \includegraphics[width=1\linewidth,height=\textheight,keepaspectratio]{images/scispace.png}

}

\caption{SciSpace: An AI Research Agent}

\end{figure}%

\subsection*{Step 4: Reading, Organizing, and
Synthesizing}\label{step-4-reading-organizing-and-synthesizing}
\addcontentsline{toc}{subsection}{Step 4: Reading, Organizing, and
Synthesizing}

Once you have gathered a core set of relevant and credible articles, the
real intellectual work begins. This is the stage where you move from
being a collector of information to a synthesizer of knowledge. To
manage this process effectively, it is essential to use
\textbf{reference management software} like Zotero, Mendeley, or EndNote
from the very start. These tools are indispensable for modern research,
allowing you to build a personal digital library where you can organize
PDFs, take systematic notes, and automatically generate citations and
bibliographies in your word processor. Adopting this practice early will
save you countless hours and prevent significant frustration down the
road.

\begin{figure}[H]

{\centering \includegraphics[width=1\linewidth,height=\textheight,keepaspectratio]{images/zotero.png}

}

\caption{Zotero: Reference Management Software}

\end{figure}%

As you read, it is critical to understand the distinction between an
annotated bibliography and a literature review. An \textbf{annotated
bibliography} is simply a list of sources, where each entry is followed
by a paragraph that summarizes that single source in isolation. A
\textbf{literature review}, by contrast, organizes ideas and findings
thematically, not by source. Think of it this way: an annotated
bibliography is a list of ingredients, while a literature review is the
finished dish, where those ingredients have been combined to create
something new. Your goal is to write the review, not the bibliography.

The heart of this process is \textbf{synthesis}---the act of weaving
together findings from different studies to create a new, integrated
understanding. This goes far beyond summary. Synthesis requires you to
read across your sources, actively looking for patterns, connections,
and discrepancies. As you read, ask yourself: Where do different authors
agree? Where do they disagree, and why? How does a finding from one
study build upon, challenge, or refine a finding from another? Your job
is to narrate this conversation, summarizing the key points and
highlighting the critical debates and tensions within the literature. A
\textbf{literature map} can be an invaluable visual tool here. By
mapping out your main themes and clustering the key studies under each
one, you can begin to see the structure of the conversation and the
relationships between different pieces of research, creating a clear
outline for your written review.

\subsection*{Step 5: Structuring and Writing the
Review}\label{step-5-structuring-and-writing-the-review}
\addcontentsline{toc}{subsection}{Step 5: Structuring and Writing the
Review}

With your synthesized notes and literature map as your guide, you are
ready to write. A literature review should not be a dry recitation of
facts but a compelling, well-structured narrative with a clear
introduction, body, and conclusion.

The \textbf{introduction} should establish the significance of the broad
research topic and provide a roadmap for the reader. It should clearly
state the scope of your review---what you will and will not be
covering---and briefly outline the major themes that you will discuss in
the body.

The \textbf{body} of the review should be organized thematically,
following the structure of your literature map. Each section or major
paragraph should focus on a specific theme or debate, beginning with a
clear topic sentence that introduces the point you are about to make.
Within each section, you must synthesize the findings from multiple
sources. Instead of dedicating a paragraph to each study, you should
make a claim and use evidence from several studies to support it (e.g.,
``Several studies have found a consistent link between X and Y\ldots{}''
or ``The debate over Z is characterized by two main schools of
thought\ldots{}''). A strong review does not ignore contradictory
findings; instead, it acknowledges and discusses these conflicts,
attempting to explain them (e.g., ``While most studies find X, Author D
(2021) found Y, possibly due to a different methodology\ldots{}''). Use
clear transitions to create a smooth, logical flow from one theme to the
next, building your argument step by step.

The entire review builds toward the \textbf{conclusion}, which is the
most essential part of the argument. First, briefly summarize the main
takeaways from the literature you have reviewed. Then, pivot to the ``so
what'' by explicitly identifying the \textbf{gap, contradiction, or
unanswered question} that your systematic review has uncovered. This is
the punchline. Finally, state the purpose of your own proposed study,
clearly explaining how it is uniquely positioned to address this
specific gap and, therefore, make a valuable and original contribution
to the scholarly conversation.

\begin{figure}[H]

{\centering \includegraphics[width=1\linewidth,height=\textheight,keepaspectratio]{images/lit-review.png}

}

\caption{Literature Review Components}

\end{figure}%

\section*{Knowing When to Stop: The Concept of
Saturation}\label{knowing-when-to-stop-the-concept-of-saturation}
\addcontentsline{toc}{section}{Knowing When to Stop: The Concept of
Saturation}

How do you know when you are done searching for literature? The guiding
principle is the concept of \textbf{saturation}. You have reached
saturation when your searches through databases and citation chains
begin to yield little to no new information. You start seeing the same
authors and the same seminal articles cited repeatedly, and any new
articles you find tend to fit neatly into the thematic categories you
have already developed in your literature map. Reaching this point of
diminishing returns is a sign that you have conducted a comprehensive
search and have a firm grasp of the scholarly literature on your topic.
It gives you the confidence to move forward with your writing, knowing
you have a solid foundation.

\section*{Conclusion: From Summary to Synthesis to Scholarly
Contribution}\label{conclusion-from-summary-to-synthesis-to-scholarly-contribution}
\addcontentsline{toc}{section}{Conclusion: From Summary to Synthesis to
Scholarly Contribution}

The literature review is far more than a preliminary chore; it is a
foundational and intellectually rigorous part of the research process
itself. It is the mechanism through which you join a scholarly
community, transforming yourself from a passive consumer of knowledge
into an active participant in its creation. By systematically finding,
evaluating, and synthesizing the work of others, you demonstrate your
competence as a researcher and earn the credibility needed for your own
voice to be heard. It is in the act of critically reviewing the
literature that you discover the gaps in our collective understanding
and, in doing so, find the precise space where your unique contribution
can and should be made. A well-crafted literature review is, therefore,
not just a summary of what is known; it is a persuasive argument for
what needs to be known next.

\section*{Journal Prompts}\label{journal-prompts}
\addcontentsline{toc}{section}{Journal Prompts}

\begin{enumerate}
\def\labelenumi{\arabic{enumi}.}
\item
  Reflect on the metaphor introduced at the beginning of the chapter:
  walking into a conversation that's already underway. Have you ever had
  that experience in real life (in class, online, or at work)? What
  happened when you did---or didn't---take the time to listen first? How
  does that scenario relate to the role of the literature review in
  research? Why is it important to understand what's already been said
  before adding your ideas?
\item
  Think about a media-related topic that interests you (e.g., influencer
  culture, video game violence, media portrayals of mental health). Now
  imagine you are preparing to write a literature review on that topic.
  What kind of ``gap'' would you look for to justify a new study? Would
  it be a topical void, a contradiction, or an overlooked perspective?
  Why does that kind of gap matter in media research?
\item
  In your own words, explain the difference between an annotated
  bibliography and a proper literature review. Why is that difference
  significant? Reflect on a time when you had to summarize multiple
  sources for a paper or project. Did you organize those sources
  thematically, or treat each one individually? Looking ahead, how will
  your approach change when writing your literature review?
\end{enumerate}


\backmatter


\end{document}
