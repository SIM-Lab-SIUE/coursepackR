% Options for packages loaded elsewhere
% Options for packages loaded elsewhere
\PassOptionsToPackage{unicode}{hyperref}
\PassOptionsToPackage{hyphens}{url}
%
\documentclass[
  letterpaper,
]{book}
\usepackage{xcolor}
\usepackage[margin=1in]{geometry}
\usepackage{amsmath,amssymb}
\setcounter{secnumdepth}{5}
\usepackage{iftex}
\ifPDFTeX
  \usepackage[T1]{fontenc}
  \usepackage[utf8]{inputenc}
  \usepackage{textcomp} % provide euro and other symbols
\else % if luatex or xetex
  \usepackage{unicode-math} % this also loads fontspec
  \defaultfontfeatures{Scale=MatchLowercase}
  \defaultfontfeatures[\rmfamily]{Ligatures=TeX,Scale=1}
\fi
\usepackage{lmodern}
\ifPDFTeX\else
  % xetex/luatex font selection
\fi
% Use upquote if available, for straight quotes in verbatim environments
\IfFileExists{upquote.sty}{\usepackage{upquote}}{}
\IfFileExists{microtype.sty}{% use microtype if available
  \usepackage[]{microtype}
  \UseMicrotypeSet[protrusion]{basicmath} % disable protrusion for tt fonts
}{}
\makeatletter
\@ifundefined{KOMAClassName}{% if non-KOMA class
  \IfFileExists{parskip.sty}{%
    \usepackage{parskip}
  }{% else
    \setlength{\parindent}{0pt}
    \setlength{\parskip}{6pt plus 2pt minus 1pt}}
}{% if KOMA class
  \KOMAoptions{parskip=half}}
\makeatother
% Make \paragraph and \subparagraph free-standing
\makeatletter
\ifx\paragraph\undefined\else
  \let\oldparagraph\paragraph
  \renewcommand{\paragraph}{
    \@ifstar
      \xxxParagraphStar
      \xxxParagraphNoStar
  }
  \newcommand{\xxxParagraphStar}[1]{\oldparagraph*{#1}\mbox{}}
  \newcommand{\xxxParagraphNoStar}[1]{\oldparagraph{#1}\mbox{}}
\fi
\ifx\subparagraph\undefined\else
  \let\oldsubparagraph\subparagraph
  \renewcommand{\subparagraph}{
    \@ifstar
      \xxxSubParagraphStar
      \xxxSubParagraphNoStar
  }
  \newcommand{\xxxSubParagraphStar}[1]{\oldsubparagraph*{#1}\mbox{}}
  \newcommand{\xxxSubParagraphNoStar}[1]{\oldsubparagraph{#1}\mbox{}}
\fi
\makeatother

\usepackage{color}
\usepackage{fancyvrb}
\newcommand{\VerbBar}{|}
\newcommand{\VERB}{\Verb[commandchars=\\\{\}]}
\DefineVerbatimEnvironment{Highlighting}{Verbatim}{commandchars=\\\{\}}
% Add ',fontsize=\small' for more characters per line
\usepackage{framed}
\definecolor{shadecolor}{RGB}{241,243,245}
\newenvironment{Shaded}{\begin{snugshade}}{\end{snugshade}}
\newcommand{\AlertTok}[1]{\textcolor[rgb]{0.68,0.00,0.00}{#1}}
\newcommand{\AnnotationTok}[1]{\textcolor[rgb]{0.37,0.37,0.37}{#1}}
\newcommand{\AttributeTok}[1]{\textcolor[rgb]{0.40,0.45,0.13}{#1}}
\newcommand{\BaseNTok}[1]{\textcolor[rgb]{0.68,0.00,0.00}{#1}}
\newcommand{\BuiltInTok}[1]{\textcolor[rgb]{0.00,0.23,0.31}{#1}}
\newcommand{\CharTok}[1]{\textcolor[rgb]{0.13,0.47,0.30}{#1}}
\newcommand{\CommentTok}[1]{\textcolor[rgb]{0.37,0.37,0.37}{#1}}
\newcommand{\CommentVarTok}[1]{\textcolor[rgb]{0.37,0.37,0.37}{\textit{#1}}}
\newcommand{\ConstantTok}[1]{\textcolor[rgb]{0.56,0.35,0.01}{#1}}
\newcommand{\ControlFlowTok}[1]{\textcolor[rgb]{0.00,0.23,0.31}{\textbf{#1}}}
\newcommand{\DataTypeTok}[1]{\textcolor[rgb]{0.68,0.00,0.00}{#1}}
\newcommand{\DecValTok}[1]{\textcolor[rgb]{0.68,0.00,0.00}{#1}}
\newcommand{\DocumentationTok}[1]{\textcolor[rgb]{0.37,0.37,0.37}{\textit{#1}}}
\newcommand{\ErrorTok}[1]{\textcolor[rgb]{0.68,0.00,0.00}{#1}}
\newcommand{\ExtensionTok}[1]{\textcolor[rgb]{0.00,0.23,0.31}{#1}}
\newcommand{\FloatTok}[1]{\textcolor[rgb]{0.68,0.00,0.00}{#1}}
\newcommand{\FunctionTok}[1]{\textcolor[rgb]{0.28,0.35,0.67}{#1}}
\newcommand{\ImportTok}[1]{\textcolor[rgb]{0.00,0.46,0.62}{#1}}
\newcommand{\InformationTok}[1]{\textcolor[rgb]{0.37,0.37,0.37}{#1}}
\newcommand{\KeywordTok}[1]{\textcolor[rgb]{0.00,0.23,0.31}{\textbf{#1}}}
\newcommand{\NormalTok}[1]{\textcolor[rgb]{0.00,0.23,0.31}{#1}}
\newcommand{\OperatorTok}[1]{\textcolor[rgb]{0.37,0.37,0.37}{#1}}
\newcommand{\OtherTok}[1]{\textcolor[rgb]{0.00,0.23,0.31}{#1}}
\newcommand{\PreprocessorTok}[1]{\textcolor[rgb]{0.68,0.00,0.00}{#1}}
\newcommand{\RegionMarkerTok}[1]{\textcolor[rgb]{0.00,0.23,0.31}{#1}}
\newcommand{\SpecialCharTok}[1]{\textcolor[rgb]{0.37,0.37,0.37}{#1}}
\newcommand{\SpecialStringTok}[1]{\textcolor[rgb]{0.13,0.47,0.30}{#1}}
\newcommand{\StringTok}[1]{\textcolor[rgb]{0.13,0.47,0.30}{#1}}
\newcommand{\VariableTok}[1]{\textcolor[rgb]{0.07,0.07,0.07}{#1}}
\newcommand{\VerbatimStringTok}[1]{\textcolor[rgb]{0.13,0.47,0.30}{#1}}
\newcommand{\WarningTok}[1]{\textcolor[rgb]{0.37,0.37,0.37}{\textit{#1}}}

\usepackage{longtable,booktabs,array}
\usepackage{calc} % for calculating minipage widths
% Correct order of tables after \paragraph or \subparagraph
\usepackage{etoolbox}
\makeatletter
\patchcmd\longtable{\par}{\if@noskipsec\mbox{}\fi\par}{}{}
\makeatother
% Allow footnotes in longtable head/foot
\IfFileExists{footnotehyper.sty}{\usepackage{footnotehyper}}{\usepackage{footnote}}
\makesavenoteenv{longtable}
\usepackage{graphicx}
\makeatletter
\newsavebox\pandoc@box
\newcommand*\pandocbounded[1]{% scales image to fit in text height/width
  \sbox\pandoc@box{#1}%
  \Gscale@div\@tempa{\textheight}{\dimexpr\ht\pandoc@box+\dp\pandoc@box\relax}%
  \Gscale@div\@tempb{\linewidth}{\wd\pandoc@box}%
  \ifdim\@tempb\p@<\@tempa\p@\let\@tempa\@tempb\fi% select the smaller of both
  \ifdim\@tempa\p@<\p@\scalebox{\@tempa}{\usebox\pandoc@box}%
  \else\usebox{\pandoc@box}%
  \fi%
}
% Set default figure placement to htbp
\def\fps@figure{htbp}
\makeatother





\setlength{\emergencystretch}{3em} % prevent overfull lines

\providecommand{\tightlist}{%
  \setlength{\itemsep}{0pt}\setlength{\parskip}{0pt}}



 


\makeatletter
\@ifpackageloaded{caption}{}{\usepackage{caption}}
\AtBeginDocument{%
\ifdefined\contentsname
  \renewcommand*\contentsname{Table of contents}
\else
  \newcommand\contentsname{Table of contents}
\fi
\ifdefined\listfigurename
  \renewcommand*\listfigurename{List of Figures}
\else
  \newcommand\listfigurename{List of Figures}
\fi
\ifdefined\listtablename
  \renewcommand*\listtablename{List of Tables}
\else
  \newcommand\listtablename{List of Tables}
\fi
\ifdefined\figurename
  \renewcommand*\figurename{Figure}
\else
  \newcommand\figurename{Figure}
\fi
\ifdefined\tablename
  \renewcommand*\tablename{Table}
\else
  \newcommand\tablename{Table}
\fi
}
\@ifpackageloaded{float}{}{\usepackage{float}}
\floatstyle{ruled}
\@ifundefined{c@chapter}{\newfloat{codelisting}{h}{lop}}{\newfloat{codelisting}{h}{lop}[chapter]}
\floatname{codelisting}{Listing}
\newcommand*\listoflistings{\listof{codelisting}{List of Listings}}
\makeatother
\makeatletter
\makeatother
\makeatletter
\@ifpackageloaded{caption}{}{\usepackage{caption}}
\@ifpackageloaded{subcaption}{}{\usepackage{subcaption}}
\makeatother
\usepackage{bookmark}
\IfFileExists{xurl.sty}{\usepackage{xurl}}{} % add URL line breaks if available
\urlstyle{same}
\hypersetup{
  pdftitle={Introduction to R and RStudio},
  hidelinks,
  pdfcreator={LaTeX via pandoc}}


\title{Introduction to R and RStudio}
\author{}
\date{}
\begin{document}
\frontmatter
\maketitle

\renewcommand*\contentsname{Table of contents}
{
\setcounter{tocdepth}{2}
\tableofcontents
}

\mainmatter
R is a free, open-source language for \textbf{data manipulation,
analysis, and visualization}. RStudio is the \textbf{IDE} (workbench)
that makes R friendlier and faster. Together they give you a modern
workflow for \textbf{reproducible research} in mass communication.

Use R for the statistics and graphics; use \textbf{RStudio} to write
code, render reports, manage files, and keep projects organized.

\begin{center}\rule{0.5\linewidth}{0.5pt}\end{center}

\chapter{What is R?}\label{what-is-r}

\begin{itemize}
\tightlist
\item
  A programming language for statistics and data work (from simple
  summaries to machine learning).
\item
  Script-based: you \textbf{write code} that documents exactly what you
  did---great for transparency and reuse.
\item
  Superpowers via \textbf{packages} (e.g., \texttt{tidyverse} for
  wrangling, \texttt{ggplot2} for visualization).
\end{itemize}

\textbf{Why it matters here:} You'll analyze surveys, scrape or analyze
social media, and make publication-quality figures---repeatably.

\begin{center}\rule{0.5\linewidth}{0.5pt}\end{center}

\chapter{What is RStudio?}\label{what-is-rstudio}

RStudio (by Posit) is the \textbf{integrated development environment}
for R:

\begin{itemize}
\tightlist
\item
  \textbf{Source editor} with syntax highlighting and autocomplete
\item
  \textbf{Console} for running code
\item
  \textbf{Environment} to inspect objects/datasets
\item
  \textbf{Files/Plots/Packages/Help/Viewer} to manage outputs and docs
\item
  Tight integration with \textbf{Quarto} (dynamic documents) and
  \textbf{Git}
\end{itemize}

\textbf{Outcome:} faster iteration, cleaner organization, and easier
collaboration.

\begin{center}\rule{0.5\linewidth}{0.5pt}\end{center}

\chapter{Why Use R + RStudio?}\label{why-use-r-rstudio}

\section{Open Source}\label{open-source}

Free to install and extend. A huge community keeps adding new methods
without license fees.

\section{Analysis \& Visualization}\label{analysis-visualization}

From descriptive stats to regression and beyond in one place;
\texttt{ggplot2} produces clear, customizable graphics.

\section{Reproducible Research}\label{reproducible-research}

Quarto documents combine \textbf{text + code + output} in one file.
Re-render to update everything automatically.

\begin{figure}[H]

{\centering \includegraphics[width=1\linewidth,height=\textheight,keepaspectratio]{images/reproducibility.png}

}

\caption{Reproducibility Spectrum}

\end{figure}%

\section{Flexible \& Extensible}\label{flexible-extensible}

Thousands of packages for text analysis, social media data, networks,
etc. Write your own functions when needed.

\begin{center}\rule{0.5\linewidth}{0.5pt}\end{center}

\chapter{Install R and RStudio}\label{install-r-and-rstudio}

\section{1) Install R}\label{install-r}

\begin{enumerate}
\def\labelenumi{\arabic{enumi}.}
\tightlist
\item
  Go to \url{https://cran.r-project.org/}
\item
  Choose your OS (Windows / macOS / Linux) and install.
\end{enumerate}

\section{2) Install RStudio}\label{install-rstudio}

\begin{enumerate}
\def\labelenumi{\arabic{enumi}.}
\tightlist
\item
  Go to \url{https://posit.co/download/rstudio-desktop/}
\item
  Download \textbf{RStudio Desktop (free)} for your OS and install.
\end{enumerate}

Install \textbf{R first}, then \textbf{RStudio}. RStudio detects your R
installation at launch.

\begin{center}\rule{0.5\linewidth}{0.5pt}\end{center}

\chapter{Meet the RStudio Interface (Four
Panes)}\label{meet-the-rstudio-interface-four-panes}

\begin{itemize}
\tightlist
\item
  \textbf{Source (top-left):} write/edit scripts (\texttt{.R}) and
  Quarto files (\texttt{.qmd}).
\item
  \textbf{Console (bottom-left):} run code interactively; see
  results/errors.
\item
  \textbf{Environment (top-right):} objects in memory (data frames,
  models, etc.).
\item
  \textbf{Output (bottom-right):} Files, Plots, Packages, Help, Viewer
  (for HTML/Shiny/Quarto previews).
\end{itemize}

\begin{quote}
For a deeper tour, see \textbf{RStudio's Four Panes} in this section.
\end{quote}

\begin{center}\rule{0.5\linewidth}{0.5pt}\end{center}

\chapter{Start a New Project}\label{start-a-new-project}

Projects keep everything for one assignment/research task in one folder.

\begin{enumerate}
\def\labelenumi{\arabic{enumi}.}
\tightlist
\item
  \textbf{File → New Project\ldots{}}
\item
  Choose \textbf{New Directory → New Project} (or link an existing
  folder).
\item
  Name it and create.
\end{enumerate}

Benefits: consistent working directory, clean file paths, and fewer
``where did that file go?'' moments. Git can be enabled during setup for
version control.

\begin{center}\rule{0.5\linewidth}{0.5pt}\end{center}

\chapter{File Management Essentials}\label{file-management-essentials}

\section{R Script vs.~R Markdown /
Quarto}\label{r-script-vs.-r-markdown-quarto}

\begin{itemize}
\tightlist
\item
  \textbf{R Script (\texttt{.R})}: pure code; great for fast analysis.
\item
  \textbf{Quarto (\texttt{.qmd})}: prose + code + output → renders to
  HTML/PDF/Word for reports.
\end{itemize}

\begin{Shaded}
\begin{Highlighting}[]
\CommentTok{\# R Script example}
\FunctionTok{summary}\NormalTok{(cars)}
\FunctionTok{plot}\NormalTok{(cars)}
\end{Highlighting}
\end{Shaded}

\begin{Shaded}
\begin{Highlighting}[]
\CommentTok{\# Quarto example chunk (runs when you render the doc)}
\FunctionTok{library}\NormalTok{(ggplot2)}
\FunctionTok{ggplot}\NormalTok{(cars, }\FunctionTok{aes}\NormalTok{(speed, dist)) }\SpecialCharTok{+} \FunctionTok{geom\_point}\NormalTok{()}
\end{Highlighting}
\end{Shaded}

\pandocbounded{\includegraphics[keepaspectratio]{r-rstudio-intro-legacy_files/figure-pdf/unnamed-chunk-1-1.pdf}}

\section{CSV vs.~Excel}\label{csv-vs.-excel}

\begin{itemize}
\tightlist
\item
  Prefer \textbf{CSV} for clean, durable data.
\item
  Use \textbf{Excel} only when collaborators require it or you genuinely
  need multiple sheets.
\end{itemize}

\section{Suggested subfolders}\label{suggested-subfolders}

\begin{verbatim}
data/     # raw and cleaned datasets
scripts/  # R scripts
reports/  # .qmd / rendered outputs
output/   # figures, tables, exports
\end{verbatim}

\begin{center}\rule{0.5\linewidth}{0.5pt}\end{center}

\chapter{Package Management}\label{package-management}

\section{Install once, load each
session}\label{install-once-load-each-session}

\begin{Shaded}
\begin{Highlighting}[]
\FunctionTok{install.packages}\NormalTok{(}\StringTok{"ggplot2"}\NormalTok{)  }\CommentTok{\# once}
\FunctionTok{library}\NormalTok{(ggplot2)             }\CommentTok{\# each new R session}
\end{Highlighting}
\end{Shaded}

\section{Common packages for this
course}\label{common-packages-for-this-course}

\begin{itemize}
\tightlist
\item
  \textbf{\texttt{tidyverse}}: wrangling + plotting
\item
  \textbf{\texttt{ggplot2}}: visualization
\item
  \textbf{\texttt{dplyr}}: data manipulation
\item
  \textbf{\texttt{quanteda} / \texttt{tm}}: text analysis
\item
  \textbf{\texttt{rtweet}}: Twitter/X data (when permitted)
\end{itemize}

Update periodically:

\begin{Shaded}
\begin{Highlighting}[]
\FunctionTok{update.packages}\NormalTok{(}\AttributeTok{ask =} \ConstantTok{FALSE}\NormalTok{)}
\end{Highlighting}
\end{Shaded}

\begin{center}\rule{0.5\linewidth}{0.5pt}\end{center}

\chapter{Basics of R (Quick Start)}\label{basics-of-r-quick-start}

\section{Arithmetic}\label{arithmetic}

\begin{Shaded}
\begin{Highlighting}[]
\DecValTok{5} \SpecialCharTok{+} \DecValTok{3}     \CommentTok{\# 8}
\DecValTok{5} \SpecialCharTok{{-}} \DecValTok{3}     \CommentTok{\# 2}
\DecValTok{5} \SpecialCharTok{*} \DecValTok{3}     \CommentTok{\# 15}
\DecValTok{5} \SpecialCharTok{/} \DecValTok{3}     \CommentTok{\# 1.6667}
\DecValTok{5} \SpecialCharTok{\^{}} \DecValTok{3}     \CommentTok{\# 125}
\DecValTok{5} \SpecialCharTok{\%\%} \DecValTok{3}    \CommentTok{\# 2 (modulus)}
\end{Highlighting}
\end{Shaded}

\section{Variables}\label{variables}

\begin{Shaded}
\begin{Highlighting}[]
\NormalTok{x }\OtherTok{\textless{}{-}} \DecValTok{10}          \CommentTok{\# preferred assignment in R}
\NormalTok{y }\OtherTok{=} \DecValTok{20}           \CommentTok{\# also works}
\NormalTok{name }\OtherTok{\textless{}{-}} \StringTok{"Alex"}
\end{Highlighting}
\end{Shaded}

\section{Functions}\label{functions}

\begin{Shaded}
\begin{Highlighting}[]
\FunctionTok{sum}\NormalTok{(}\DecValTok{1}\NormalTok{, }\DecValTok{2}\NormalTok{, }\DecValTok{3}\NormalTok{)            }\CommentTok{\# 6}
\FunctionTok{mean}\NormalTok{(}\FunctionTok{c}\NormalTok{(}\DecValTok{1}\NormalTok{, }\DecValTok{2}\NormalTok{, }\DecValTok{3}\NormalTok{, }\DecValTok{4}\NormalTok{))     }\CommentTok{\# 2.5}
\FunctionTok{sqrt}\NormalTok{(}\DecValTok{16}\NormalTok{)                }\CommentTok{\# 4}
\end{Highlighting}
\end{Shaded}

\begin{center}\rule{0.5\linewidth}{0.5pt}\end{center}

\chapter{Commenting \& Organizing
Code}\label{commenting-organizing-code}

\section{Comments}\label{comments}

\begin{Shaded}
\begin{Highlighting}[]
\CommentTok{\# This is a comment for humans}
\NormalTok{age }\OtherTok{\textless{}{-}} \FunctionTok{c}\NormalTok{(}\DecValTok{18}\NormalTok{, }\DecValTok{23}\NormalTok{, }\DecValTok{21}\NormalTok{, }\DecValTok{30}\NormalTok{)   }\CommentTok{\# vector of ages}
\end{Highlighting}
\end{Shaded}

\section{Sections}\label{sections}

\begin{Shaded}
\begin{Highlighting}[]
\CommentTok{\# =========================}
\CommentTok{\# Section: Data Preparation}
\CommentTok{\# =========================}
\end{Highlighting}
\end{Shaded}

In Quarto, use markdown headings to structure your narrative:

\begin{Shaded}
\begin{Highlighting}[]
\FunctionTok{\#\# Data Preparation}

\NormalTok{Here we clean variables and handle missing values.}
\end{Highlighting}
\end{Shaded}

Keep code tidy: consistent indentation, small steps, and meaningful
names. Your future self (and collaborators) will thank you.

\begin{center}\rule{0.5\linewidth}{0.5pt}\end{center}

\chapter{First Run: Your Hello World}\label{first-run-your-hello-world}

\begin{Shaded}
\begin{Highlighting}[]
\FunctionTok{print}\NormalTok{(}\StringTok{"Hello, World!"}\NormalTok{)}
\end{Highlighting}
\end{Shaded}

Open a Quarto file and add a code chunk:

\begin{verbatim}

::: {.cell}

```{.r .cell-code}
print("Hello, World from Quarto!")
```

::: {.cell-output .cell-output-stdout}

```
[1] "Hello, World from Quarto!"
```


:::
:::
\end{verbatim}

Render the document to see executable output embedded in your report.

\begin{center}\rule{0.5\linewidth}{0.5pt}\end{center}

\chapter{Next Steps}\label{next-steps}

\begin{itemize}
\tightlist
\item
  Install the \textbf{course package} and pull your \textbf{Journal}
  scaffold.
\item
  Read \textbf{RStudio's Four Panes} and \textbf{Global Options} pages
  to customize your workspace.
\item
  Commit your project to \textbf{GitHub} and set your \textbf{Profile
  README}.
\end{itemize}


\backmatter


\end{document}
